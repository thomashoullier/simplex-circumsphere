\documentclass[]{article}

\usepackage{stmaryrd}
\usepackage{amsmath}

\title{n-Simplex circumsphere}
\author{Thomas HOULLIER}

\begin{document}
\maketitle

\begin{abstract}
We provide a short note on the computation of a n-simplex circumsphere. This is
adapted from a blog post by Gautam Manohar and another blog post by G.
Westendorp. Please note that I am not a mathematician and that I have not
checked in depth the claims I make.
\end{abstract}

Let:
\begin{itemize}
\item $n$ the dimensionality
\item A simplex with $n+1$ vertices
\item The vertices $V_i$, $i\in \llbracket 0, n \rrbracket$
\item The coordinates of vertex $V_i$: $V_i^j$, $j\in\llbracket 1, n \rrbracket$
\item The circumsphere with center $C$, with coordinates $C^j$, and radius $R$
\end{itemize}

\section{Method 1: Circumsphere equation}
Any vertex $V_i$ is on the circumsphere, so $\forall i$, it obeys eq
(\ref{vertex-on-sphere}).

\begin{equation} \label{vertex-on-sphere}
\sum_j \left( V_i^j - C^j \right)^2 = R^2
\end{equation}

Arbitrarily, we take the first equation, for the first vertex $i=0$, and
subtract it from all the others. Then $\forall i \in \llbracket 1, n
\rrbracket$, we have eq (\ref{row-equations}) after some expansion and
cancellation of terms.

\begin{equation} \label{row-equations}
2 \sum_j \left( V_0^j - V_i^j \right) \cdot C^j =
  \sum_j \left( V_0^j \right)^2 - \sum_j \left( V_i^j \right)^2
\end{equation}

This is readily expressed in matrix format, with each $i$ being a row
in a linear system of equations (\ref{matrix-system}).

\begin{equation} \label{matrix-system}
A \cdot x = b
\end{equation}

With:
\begin{itemize}
\item $A$ a square matrix with elements $A_{i,j}$ (\ref{a})
\item $x$ the vector of $C^j$ coordinates.
\item $b$ a right hand side vector defined by eq (\ref{b})
\end{itemize}

\begin{equation} \label{a}
A_{i,j} = 2 \left( V_0^j - V_i^j \right)
\end{equation}

\begin{equation} \label{b}
b_{i} = \sum_j \left( V_0^j \right)^2 - \sum_j \left( V_i^j \right)^2
\end{equation}

We can solve for $x = C$ using any linear equation solver. Next we can
retrieve $R$ from the distance of any vertex from the center, for example
eq (\ref{radius}).

\begin{equation} \label{radius}
R = \sqrt{\sum_j \left( V_0^j - C^j \right)^2}
\end{equation}

Thus, we have computed the center $C$ of the circumsphere, and $R$ its radius.

\section{Method 2: Cayley-Menger matrix}
As noted by Westendorp, building on a work by Coxeter (please refer to his
blog post for the proof). We can compute the circumsphere of a simplex
with the aid of the Cayley-Menger matrix. The Cayley-Menger matrix is
given in eq \ref{cm}.

\begin{equation} \label{cm}
CM = \begin{bmatrix}
0 & 1 & 1 & 1 & \dots \\
1 & 0 & d_0^1 & d_0^2 & \dots \\
1 & d_1^0 & 0 & d_1^2 & \dots \\
1 & d_2^0 & d_2^1 & 0 & \dots \\
\vdots & \vdots & \vdots & \vdots & \ddots
\end{bmatrix}
\end{equation}

with $d_i^j = \sum_k \left( V_i^k - V_j^k \right)^2$, the squared distance
between vertices $V_i$ and $V_j$.

Then, the radius of the circumsphere is given by eq \ref{cm-radius}, the
circumcenter by eq \ref{cm-coords} (please refer to the blog post by Westendorp
for the proof).

\begin{equation} \label{cm-radius}
R = \sqrt{- \frac{1}{2} \cdot {CM^{-1}}_{1,1}}
\end{equation}

\begin{equation} \label{cm-coords}
x_j = {CM^{-1}}_{1,j+1}
\end{equation}

Note that we only need the first row of the inverse $CM$ matrix, not the whole
inverse. Also note that $CM$ is symmetric, which enables simpler decompositions
for the inversion operation. $CM$ is \emph{indefinite} (proof omitted).
Since $CM$ is square symmetric, but not positive-definite, the most efficient
decomposition to perform the inverse, to the best of my knowledge is the
Bunch–Kaufman decomposition, which is only slightly better than LUP.

\section{Method tradeoff}
Starting from a list of vertices coordinates $V_i^j$, which method is the most
efficient computationally? Let's count operations to make sure.

\begin{itemize}
\item Method 1: \begin{itemize}
\item $A$: $n^2$ subtractions
\item $b$: $n(n+1)$ squares, $(n-1)(n+1)$ sums, $n$ subtractions
\item LUP decomposition and solve.
\end{itemize}
\item Method 2: \begin{itemize}
\item Distance squared for one pair of vertices: $(n-1)$ subtractions, $(n-1)$
      squares, $(n-1)$ sums.
\item Number of unique pairs with non-zero distance: $n(n-1)/2$
\item Bunch-Kaufman decomposition and solve.
\end{itemize} \end{itemize}

Long story short, the setup for the first method is $\mathcal{O}(n^2)$ in time,
the second method is $\mathcal{O}(n^3)$. Both decompositions are
$\mathcal{O}(n^3)$ with a slight advantage for Bunch-Kaufman. Method 1 is
preferable.
\end{document}
