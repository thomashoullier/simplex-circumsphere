\documentclass[]{article}

\usepackage{stmaryrd}

\title{n-Simplex circumsphere}
\author{Thomas HOULLIER}

\begin{document}
\maketitle

\begin{abstract}
We provide a short note on the computation of a n-simplex circumsphere. This is
adapted from a blog post by Gautam Manohar.
\end{abstract}

Let:
\begin{itemize}
\item $n$ the dimensionality
\item A simplex with $n+1$ vertices
\item The vertices $V_i$, $i\in \llbracket 0, n \rrbracket$
\item The coordinates of vertex $V_i$: $V_i^j$, $j\in\llbracket 1, n \rrbracket$
\item The circumsphere with center $C$, with coordinates $C^j$, and radius $R$
\end{itemize}

Any vertex $V_i$ is on the circumsphere, so $\forall i$, it obeys eq
(\ref{vertex-on-sphere}).

\begin{equation} \label{vertex-on-sphere}
\sum_j \left( V_i^j - C^j \right)^2 = R^2
\end{equation}

Arbitrarily, we take the first equation, for the first vertex $i=0$, and
subtract it from all the others. Then $\forall i \in \llbracket 1, n
\rrbracket$, we have eq (\ref{row-equations}) after some expansion and
cancellation of terms.

\begin{equation} \label{row-equations}
2 \sum_j \left( V_0^j - V_i^j \right) \cdot C^j =
  \sum_j \left( V_0^j \right)^2 - \sum_j \left( V_i^j \right)^2
\end{equation}

This is readily expressed in matrix format, with each $i$ being a row
in a linear system of equations (\ref{matrix-system}).

\begin{equation} \label{matrix-system}
A \cdot x = b
\end{equation}

With:
\begin{itemize}
\item $A$ a square matrix with elements $A_{i,j}$ (\ref{a})
\item $x$ the vector of $C^j$ coordinates.
\item $b$ a right hand side vector defined by eq (\ref{b})
\end{itemize}

\begin{equation} \label{a}
A_{i,j} = 2 \left( V_0^j - V_i^j \right)
\end{equation}

\begin{equation} \label{b}
b_{i} = \sum_j \left( V_0^j \right)^2 - \sum_j \left( V_i^j \right)^2
\end{equation}

We can solve for $x = C$ using any linear equation solver. Next we can
retrieve $R$ from the distance of any vertex from the center, for example
eq (\ref{radius}).

\begin{equation} \label{radius}
R = \sqrt{\sum_j \left( V_0^j - C^j \right)^2}
\end{equation}

Thus, we have computed the center $C$ of the circumsphere, and $R$ its radius.

\end{document}
